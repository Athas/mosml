% manual.tex v. 2.00.1 Copyright (C) Peter Sestoft 1994, 2000-04-29
%
% You may edit for lay-out, or leave out irrelevant sections (if
% such omissions are marked somehow), but you may not redistribute the
% sources.  The authors' names and the Moscow ML URL must be left in place.

\documentclass[fleqn]{article}

\usepackage{a4,isolatin1,notesart,a4wide,pslatex}
\usepackage[T1]{fontenc}

%\pdfcompresslevel=9

\begin{document}

\begin{center}

  {\huge\bf Moscow ML Owner's Manual}\\[0.5cm]

Version 2.00 of June 2000\\[0.7cm]

Sergei Romanenko, Russian Academy of Sciences, Moscow, Russia\\
Claudio Russo, Cambridge University, Cambridge, United Kingdom\\
Peter Sestoft, Royal Veterinary and Agricultural University, 
Copenhagen, Denmark

%{\large Sergei Romanenko\footnote{
%Keldysh Institute of Applied Mathematics, 
%Russian Academy of Sciences, Miusskaya Pl. 4, 125047 Moscow, Russia.  
%Email: roman@keldysh.ru}
%and Peter Sestoft\footnote{Department of Mathematics and
%  Physics, Royal Veterinary and Agricultural University,
%  Thorvaldsensvej 40, DK-1871 Frederiksberg C, Denmark.  E-mail:
%  sestoft@dina.kvl.dk}}\\[1ex]
\end{center}

\vspace{0.5cm}

\noindent 
Moscow ML implements Standard ML (SML), as defined in the 1997 {\em
  Definition of Standard ML\/}, including the SML Modules languages
and some extensions.  Moreover, Moscow ML supports most required parts
of the SML Basis Library.  It supports separate compilation and the
generation of stand-alone executables.

This document explains how to use the Moscow ML system.  A companion
document, the {\em Moscow ML Language Overview\/}, summarizes Moscow
ML syntax and the most common built-in functions
\cite{MoscowML:2000:MoscowMLLanguage}.  Another companion document,
the \emph{Moscow ML Library Documentation}, describes in detail all
Moscow ML library structures \cite{MoscowML:2000:MoscowMLLibrary}. 

{\bf Acknowledgements:} The Caml Light system was instrumental in
creating Moscow ML, which uses its runtime system and essentially the
same bytecode generator.  Many other aspects of the design were
derived from Caml Light, developed by Xavier Leroy and Damien Doligez
at INRIA, France \cite{Leroy:1990:TheZinc,Leroy:1993:TheCaml}.  The ML
Kit helped solving problems of parsing, infix resolution, and type
inference \cite{Birkedal:1993:TheML}.

\vfill

\begin{center}
\begin{tabular}{|c|}\hline
\rule[-0.4cm]{0cm}{1cm}The Moscow ML home page is\ \
    \verb$http://www.dina.kvl.dk/~sestoft/mosml.html$\\\hline
\end{tabular}
\end{center}

\vfill

\tableofcontents

\newpage

\section{Getting started}
\label{sec-getting-started}

\subsection{Installation}

Get a copy of the Moscow ML system executables (see
Section~\ref{sec-how-to-get} for instructions) and unpack them in your
home directory (under Unix) or in directory \verb#C:\# (under MS
Windows and DOS)\@.  This creates a directory {\tt mosml}.  Read the
file {\tt mosml/install.txt}.  This manual, and the {\em Moscow ML
  Language Overview\/}, are in directory {\tt mosml/doc}.


\subsection{The interactive system}

The interactive system is invoked by typing {\tt mosml} at the shell
prompt.  It allows you to enter declarations and evaluate expressions:

\begin{program}
\$ mosml
Moscow ML version 2.00 (June 2000)
Enter `quit();' to quit.
-  fun fac n = if n = 0 then 1 else n * fac (n-1);
> val fac = fn : int -> int
-  fac 10;
> val it = 3628800 : int
-
\end{program}

\noindent You can quit the interactive session by typing `{\tt
  quit();}' or control-D (under Unix) or control-Z followed by newline
(under MS Windows and DOS)\@.  Type {\tt help "lib";} for an overview
of built-in function libraries, and e.g.\ {\tt help "Array"} for help
on {\tt Array} operations.  See Section~\ref{sec-interactive-system}
for further information on {\tt mosml}.


\subsection{The batch compiler and linker}

The batch compiler and linker is invoked by typing {\tt mosmlc} at the
shell prompt.  It can compile ML source files separately ({\tt mosmlc
  -c}) and link them to obtain executables ({\tt mosmlc -o}), in a
manner similar to C compilers.  See Section~\ref{sec-batch-compiler}
for further information on {\tt mosmlc}.


\subsection{A simple module system}

Moscow ML provides a simple subset of the Standard ML Modules
language, restricted to signatures and non-nested structures.  A
Moscow ML program consists of one or more units.  A unit U has a
signature (or interface) in file U.sig and a body (or implementation)
in file U.sml.  The unit signature corresponds to a Standard ML
signature, and the unit body corresponds to a Standard ML structure.
Moscow ML supports type-safe separate compilation and linking.
Section~\ref{sec-compilation-units} gives the syntax and an informal
semantics of compilation units.  Section~\ref{sec-recompilation-guide}
explains automatic recompilation management.


\subsection{What is new in release 2.00}

\begin{itemize}
\item The full SML Modules language (structures, signatures, and
  functors) is now supported, thanks to Claudio Russo.
\item Several extensions to the SML Modules language are provided,
  thanks to Claudio Russo:
  \begin{itemize}
  \item higher-order functors: functors may be defined within
    structures and functors; see Section~\ref{sec-ho-modules}
  \item first-class modules: structures may be packed and then handled
    as Core language values, which may then be unpacked to structures
    again; see Section~\ref{sec-ho-modules}
  \item recursive modules: signatures and structures may be
    recursively defined; see Section~\ref{sec-rec-modules}
  \end{itemize}
\item Value polymorphism has become friendlier: non-generalizable free
  type variables are left free, and become instantiated (once only)
  when the bound variable is used; see Section~\ref{sec-valuepoly}.
\item Added facilities for creating and communicating with
  subprocesses (structure \texttt{Unix} and \texttt{Signal} from SML
  Basis Library).
\item Added facilities for efficient functional generation of HTML
  code (structure \texttt{Msp}); also supports the writing of ML
  Server Page scripts.
\item Added facilities setting and accessing `cookies' in CGI scripts
  \texttt{Mosmlcookie}, thanks to Hans Molin, Uppsala, Sweden.
\item The \texttt{Gdimage} structure now produces PNG images (using
  Thomas Boutell's gd library).
\end{itemize}

% \subsection{What is new in release 1.44}

% \begin{itemize}
% \item Added interface to the PostgreSQL and MySQL database servers
%   (structure {\tt Postgres} and {\tt Mysql}); see Sections
%   \ref{sec-using-mpq} and~\ref{sec-using-mmysql}.  Requires
%   {\tt Dynlib}.
  
% \item Added interface to POSIX 1003.2 regular expressions (structure
%   {\tt Regex}); see Section \ref{sec-using-mregex}.  Requires {\tt
%     Dynlib}.
  
% \item Added interface to Thomas Boutell's gd library for creating GIF
%   images (structure {\tt Gdimage}); see Section \ref{sec-using-mgd}.
%   Requires {\tt Dynlib}.
  
% \item Added interface to Internet and file sockets (structure {\tt
%     Socket}); see Section \ref{sec-using-msocket}. Requires {\tt
%     Dynlib}.
  
% \item Added registration of ML values, including functions, for access
%   from C code (structure \texttt{Callback}).
  
% \item Faster bytecode execution, especially on RISC CPUs.
  
% \item A list of the loaded structures is accessible as {\tt
%     Meta.loaded()}.

% \item For other minor changes and fixes, see file
%   mosml/doc/releases.txt.
% \end{itemize}


% \subsection{What is new in release 1.43}

% \begin{itemize}

% \item Weak pointers and arrays of weak pointers (structure Weak); see
%   Section~\ref{sec-weak-pointers}.

% \item The load paths can be set from the interactive system, and the
%   system's prompts and responses can be turned off (option {\tt
%     -quietdec}, variable {\tt Meta.quietdec}).  This facilitates
%   writing scripts with mosml.

% \item Prettyprinters can be installed also on base types and abstract
%   types.

% \item  The Help facility can be adapted to other uses.

% \item  Mosmllex now supports abbreviations for regular expressions
%   (thanks to Ken Larsen).

% \item Added dynamic linking of external functions (structure {\tt
%     Dynlib}) under Linux, Solaris and OSF/1 (thanks to Ken Larsen).
%   See Section~\ref{sec-dynamic-linking}.

% \item Access to GNU gdbm persistent hashtables (structures {\tt Gdbm},
%   {\tt Polygdbm}); see Section~\ref{sec-using-mgdbm}.  Requires {\tt
%     Dynlib}.

% \item For other minor changes and fixes, see file
%   mosml/doc/releases.txt.
% \end{itemize}

%\subsection{What was new in release 1.42}
%
%\begin{itemize}
%\item The linker ({\tt mosml -o A.uo B.uo \ldots}) now automatically
%  includes all (and only) those bytecode files referred to by the
%  specified files {\tt A.uo}, {\tt B.uo}, etc.  Option {\tt -i} makes
%  the linker report which files were linked.  Option {\tt -noautolink}
%  disables the autolinker and gives you full control of the order of
%  linking.
%
%\item A new structure Mosmlcgi provides support for writing CGI
%  scripts in Moscow ML (thanks to Jonas Barklund).
%
%\item Structures Array2, CommandLine, and Option have been added to
%  the Basis Library.  Structure Old was renamed to SML90.  
%
%\item Faster compilation, thanks to changes in compiler internals.
%
%\item Faster bytecode execution, thanks to Doug Currie.
%
%\item Explicit type variable parameters (introduced in SML'96) in val
%  and fun declarations have been implemented.
%
%\item For other minor changes and fixes, see file
%  mosml/doc/releases.txt.
%\end{itemize}
%
%
%\subsection{What was new in release 1.41}
%
%\begin{itemize}
%\item The input-output functions are no longer at top-level; they are
%  now in units {\tt TextIO} and {\tt BinIO}.  The math functions are
%  no longer at top-level; they are in {\tt Math}.
%
%\item The type checker now uses value polymorphism by default,
%  avoiding the mysterious imperative type variables \verb#'_a#; see
%  Section~\ref{sec-valuepoly}.
%
%\item If your existing program fails to compile because of these
%  changes, try invoking {\tt mosml} with options {\tt -imptypes -P nj93}.
%
%\item Overloaded operators are resolved to type {\tt int} by default.
%  Hence {\tt fn (x, y) => x + y} now elaborates, with type {\tt int *
%    int -> int}.
%
%\item Hexadecimal integer constants {\tt 0x1F} and word constants {\tt
%    0w31} and {\tt 0wx1f} have been introduced.  Word constants are
%  overloaded on type {\tt Word.word} and {\tt Word8.word} and resolved
%  to {\tt Word.word} by default.
%
%\item The standard syntax for signatures and non-nested structures may
%  be used in unit signatures and bodies; this improves portability
%  between Moscow ML and SML/NJ\@.  See
%  Section~\ref{sec-compilation-units}.
%\end{itemize}


%\subsection{What was new in release 1.31}
%
%\begin{itemize}
%\item Installable prettyprinters are provided.
%\end{itemize}
%
%
%\subsection{What was new in release 1.30}
%
%\begin{itemize}
%\item The system has been rewritten in Standard ML and can compile
%  itself.  This makes installation under Unix much easier than
%  it used to be.
%
%\item The garbage collector of the Caml Light runtime system has been
%  improved (by Damien Doligez at INRIA).
%
%\item The new initial basis for Standard ML has been adopted.  
%  A new primitive type
%  {\tt char} has been introduced, the types of {\tt chr}, {\tt
%    explode}, {\tt implode} and {\tt ord} have changed, and several
%  new utility libraries have been added.
%
%\item Quotations and antiquotations are supported.
%\end{itemize}

\newpage

\section{Overview and documentation}

Moscow ML implements Standard ML as revised in 1997, also known as
SML'97 \cite{Milner:1997:TheDefinition, Milner:1990:TheDefinition},
and some extensions to the Modules language.  

The {\em Moscow ML Language Overview\/}
\cite{MoscowML:2000:MoscowMLLanguage} describes the source syntax
(that is, SML'97 plus some extensions), and the most common built-in
functions.

The \emph{Moscow ML Library Documentation}
\cite{MoscowML:2000:MoscowMLLibrary} describes in detail all Moscow ML
library structures and has a comprehensive index to exceptions,
functions, structures, types, and values.  The same information is
available also from {\tt mosml}'s built-in help function
(Section~\ref{sec-mosml-help}) and as hypertext from Moscow ML's
homepage and in the mosml/doc/mosmllib/index.html.

Moscow ML implements much of the Standard ML Basis Library
\cite{SMLBasis:1996:StandardML}, the most important omission being the
functional stream input-output operations.  The Standard ML Basis
Library is a joint effort of the Standard ML of New Jersey, MLWorks,
and Moscow ML developers\footnote{The Basis Library authors are Andrew
  Appel (Princeton, USA); Emden Gansner (AT\&T Research, USA); John
  Reppy, Lal George, Lorenz Huelsbergen, Dave MacQueen (Bell
  Laboratories, USA); Matthew Arcus, Dave Berry, Richard Brooksby,
  Nick Barnes, Brian Monahan, Jon Thackray (Harlequin Ltd., Cambridge,
  England); Carsten M{\"u}ller (Berlin, Germany); and Peter Sestoft
  (Royal Veterinary and Agricultural University, Denmark).} to enhance
the portability of Standard ML programs.

The Apple Macintosh version of Moscow ML is described in detail in
\emph{Moscow ML for the Apple Macintosh}
\cite{Currie:1999:MoscowMLMac}.

%\newpage

\section{The interactive system}
\label{sec-interactive-system}

The interactive system {\tt mosml} is invoked simply by typing {\tt
  mosml} at the command line:

\begin{program}
\$ mosml
Moscow ML version 2.00 (June 2000)
Enter `quit();' to quit.
- 
\end{program}

\noindent The interactive system can be terminated by typing {\tt
  quit();} and newline, or control-D (under Unix) or control-Z and
newline (under MS Windows and DOS).  Type `{\tt help "";}' for help on
built-in functions.

Invoking the interactive system with command line arguments

\begin{program}
mosml file\sub{1} \ldots\ file\sub{n}
\end{program}

\noindent is equivalent to invoking {\tt mosml} and, when Moscow ML
has started, entering

\begin{program}
(use "file\sub{1}"; \ldots; use "file\sub{n}");  
\end{program}


\subsection{On-line help}
\label{sec-mosml-help}

In a {\tt mosml} session, you may type {\tt help "lib";} for an
overview of built-in function libraries.  To get help on a particular
identifier, such as {\tt fromString}, type

\begin{program}
help "fromstring";
\end{program}

\noindent This will produce a menu of all library structures which
contain the identifier {\tt fromstring} (disregarding the
lowercase/uppercase distinction):

{\small\begin{verbatim}
    --------------------------------
    |   1 | val  Bool.fromString   |
    |   2 | val  Char.fromString   |
    |   3 | val  Date.fromString   |
    |   4 | val  Int.fromString    |
    |   5 | val  Path.fromString   |
    |   6 | val  Real.fromString   |
    |   7 | val  String.fromString |
    |   8 | val  Time.fromString   |
    |   9 | val  Word.fromString   |
    |  10 | val  Word8.fromString  |
    --------------------------------
\end{verbatim}}

\noindent Choosing a number from this menu will invoke the help
browser on the desired structure, e.g.\ {\tt Int}.  The help browser
is primitive but easy to use.  It works best with a window size of 24
lines.


\subsection{Editing and running ML programs}
\label{sec-editing}
\begin{description}
\item[Unix and Emacs] You may run {\tt mosml} as a subshell under
  Emacs.  You should use the {\tt mosml}-version of the SML mode for
  Emacs; see file {\tt mosml/utility/emacs} for instructions.  In case
  of errors, Emacs can interpret {\tt mosml}'s error messages and jump
  to the offending piece of source code.  This is very convenient.

\item[Window systems] In a window-oriented system, such as MacOS, MS
  Windows, or the X window system, you may run {\tt mosml} in one
  window and edit source code in another.  After (re-)editing the
  source file, you must issue a {\tt use} command in the {\tt mosml}
  window.

\item[MS DOS] You may use the simple {\tt edit} script to invoke an
  editor from inside a {\tt mosml} session; see file
  \verb#mosml\utility\dosedit# for instructions.  You will not need to
  quit the {\tt mosml} session to edit a source file, and the script
  will automatically reload the newly edited file.
\end{description}


\subsection{Command-line options}
\label{sec-mosml-options}

\begin{description}
\item[{\tt -I {\rm directory}}]\mbox{ }

  Specifies directories to be searched for interface files, bytecode
  files, and source files.  A call to {\tt use}, {\tt load} or {\tt
    loadOne} will first search the current directory, then all
  directories specified by option `{\tt -I}' in order of appearance
  from left to right, and finally the standard library directory.
  (This option affects the variable {\tt Meta.loadPath}; see
  Section~\ref{sec-nonstandard-interactive}).

\item[{\tt -valuepoly}]\mbox{ }

  Specifies that the type checker should use `value polymorphism'; see
  Section~\ref{sec-valuepoly}.  Default.

\item[{\tt -imptypes}]\mbox{ }

  Specifies that the type checker should distinguish between
  imperative and applicative type variables, generalize all
  applicative type variables, and generalize imperative type variables
  only in non-expansive expressions.  See Section~\ref{sec-valuepoly}.

\item[{\tt -quietdec}]\mbox{ }

  Turns off the interactive system's prompt and responses, except for
  the two-line start-up message, warnings, and error messages.  Useful
  for writing scripts in SML\@.  Sets {\tt Meta.quietdec} to {\tt
    true}; see Section~\ref{sec-nonstandard-interactive}.

\item[{\tt -P {\rm unit-set}}]\mbox{ }

  Determines which library units will be included and open at
  compile-time.  Any library unit in the load path can be used by the
  {\tt compile} function for type checking purposes.  Thus regardless
  of the {\tt -P} option, the {\tt compile} function knows the type of
  library functions such as {\tt Array.foldl}.

  \begin{description}
  \item[{\tt -P default}] This provides an initial environment for the
    new Basis Library.  The units {\tt Array}, {\tt Char}, {\tt List},
    {\tt String}, and {\tt Vector} will be loaded, and units {\tt
      Char}, {\tt List}, and {\tt String} will be partially opened.
    This is the default.

  \item[{\tt -P sml90}] This provides an initial environment which is
    upwards compatible with that of the 1990 {\em Definition of
      Standard ML\/} and with pre-1.30 releases of Moscow ML\@.  In
    particular, the functions {\tt chr}, {\tt explode}, {\tt implode},
    and {\tt ord} work on strings, not characters.  The new versions
    of these functions are still available as {\tt Char.chr}, {\tt
      Char.ord}, {\tt String.explode}, and {\tt String.implode}.  The
    math functions and input-output facilities required by the 1990
    Definition \cite[Appendix C and D]{Milner:1990:TheDefinition} are
    available at top-level.  In addition the same (new) libraries are
    loaded as with {\tt -P default}.

  \item[{\tt -P nj93}] This provides a top-level environment which is
    mostly compatible with that of SML/NJ 0.93.  The functions {\tt
      app}, {\tt ceiling}, {\tt chr}, {\tt dec}, {\tt explode}, {\tt
      fold}, {\tt hd}, {\tt implode}, {\tt inc}, {\tt max}, {\tt min},
    {\tt nth}, {\tt nthtail}, {\tt ord}, {\tt ordof}, {\tt revapp},
    {\tt revfold}, {\tt substring}, {\tt tl}, and {\tt truncate} have
    the same type and meaning as in SML/NJ 0.93.  Note that this is
    incompatible with SML/NJ version 110\@.  The math functions and
    input-output facilities required by the 1990 Definition
    \cite[Appendix C and D]{Milner:1990:TheDefinition} are available
    at top-level.  In addition the same (new) libraries are loaded as
    with {\tt -P default}.  This option does {\em not\/} imply {\tt
      -imptypes}.

  \item[{\tt -P full}] This loads all the libraries marked {\tt F} in
    the library list (see \cite{MoscowML:1999:MoscowMLLanguage}), and
  partially opens the {\tt Char}, {\tt List}, and {\tt String} units.

  \item[{\tt -P none}] No library units are loaded or opened initially.
  \end{description}
  
  Additional library units can loaded into the interactive system by
  using the {\tt load} function; see
  Section~\ref{sec-nonstandard-interactive} below.
  
\item[{\tt -stdlib {\rm stdlib-directory}}]\mbox{ }

  Specify the standard library directory to be stdlib-directory.  The
  default standard library is usually {\tt mosml/lib} under Unix and
  \verb#c:\mosml\lib# under MS Windows and DOS.
\end{description}


\subsection{Non-standard primitives in the interactive system}
\label{sec-nonstandard-interactive}

The following non-standard primitives are defined in unit {\tt Meta},
loaded (and open by default) only in the interactive system.  Hence
these primitives cannot be used from source files which are compiled
separately.  The functions {\tt compile} and {\tt load} deal with
Moscow ML compilation units; see Section~\ref{sec-compilation-units}.

\begin{description}

\item[{\tt compile :\ string -> unit}]\mbox{ }

  Evaluating {\tt compile "{\rm U.sig}"} will compile and elaborate
  the unit signature in file U.sig, producing a compiled signature
  file U.ui.  During compilation, the compiled signatures of other
  units will be accessed if they are mentioned in U.sig.

  Evaluating {\tt compile "{\rm U.sml}"} will elaborate and compile
  the unit body in file U.sml, producing a bytecode file U.uo.  If
  there is an explicit signature U.sig, then file U.ui must exist, and
  the unit body must match the signature.  If there is no U.sig, then
  an inferred signature file U.ui will be produced also.  No
  evaluation takes place.  During compilation, the compiled signatures
  of other units will be accessed if they are mentioned in U.sml.

  The declared identifiers will be reported if {\tt verbose} is {\tt
    true} (see below); otherwise compilation will be silent.  In any
    case, compilation warnings are reported, and compilation errors
    abort the compilation and raise the exception {\tt Fail} with a
    string argument.

\item[{\tt compileStructure :\ string list -> string -> unit}]\mbox{ }

\item[{\tt compileToplevel :\ string list -> string -> unit}]\mbox{ }

\item[{\tt installPP :\ (ppstream -> 'a -> unit) -> unit}]\mbox{ }

  Evaluating {\tt installPP $pp$} installs the prettyprinter $pp$ at
  type {\tt ty}, provided $pp$ has type {\tt ppstream -> ty -> unit}.
  The type {\tt ty} must be a nullary (parameter-less) type
  constructor, either built-in (such as {\tt int} or {\tt bool}) or
  user-defined.  Whenever a value of type {\tt ty} is about to be
  printed by the interactive system, and whenever function {\tt
    printVal} is invoked on an argument of type {\tt ty}, the
  prettyprinter $pp$ will be invoked to print it.  See the example in
  {\tt mosml/examples/pretty}.

\item[{\tt load :\ string -> unit}]\mbox{ }

  Evaluating {\tt load "{\rm U}"} will load and evaluate the compiled
  unit body from file U.uo.  The resulting values are not reported,
  but exceptions are reported, and cause evaluation and loading to
  stop.  If U is already loaded, then {\tt load "{\rm U}"} has no
  effect.  If any other unit is mentioned by U but not yet loaded,
  then it will be loaded automatically before U.  The loaded unit(s)
  must be in the current directory or in a directory on the {\tt
    loadPath} list (see below).

  After loading a unit, it can be opened with {\tt open U}.  Opening
  it at top-level will list the identifiers declared in the unit.

  When loading U, it is checked that the signatures of units mentioned
  by U agree with the signatures used when compiling U, and it is
  checked that the signature of U has not been modified since U was
  compiled; these checks are necessary for type safety.  The exception
  {\tt Fail} is raised if the signature checks fail, or if the file
  containing U or a unit mentioned by U is not found.

\item[{\tt loaded :\ unit -> string list}]\mbox{ }
  
  Evaluating {\tt loaded ()} will return a list of the names of loaded
  units in some order (not including the preloaded units {\tt Meta}
  and {\tt General}).

\item[{\tt loadOne :\ string -> unit}]\mbox{ }

  Evaluating {\tt loadOne "{\rm U}"} is similar to {\tt load "{\rm
      U}"}, but raises exception {\tt Fail} if U is already loaded or
  if some unit mentioned by U is not yet loaded.  That is, it does not
  automatically load any units mentioned by U.  It performs the same
  signature checks as {\tt load}.

\item[{\tt loadPath :\ string list ref}]\mbox{ }

  This variable determines the load path: which directories will be
  searched for interface files (.ui files), bytecode files (.uo
  files), and source files (.sml files).  This variable affects the
  {\tt load}, {\tt loadOne}, and {\tt use} functions.  The current
  directory is always searched first, followed by the directories in
  {\tt loadPath}, in order.  By default, only the standard library
  directory is in the list, but if additional directories are
  specified using option {\tt -I}, then these directories are
  prepended to {\tt Meta.loadPath}.

\item[{\tt printVal :\ 'a -> 'a}]\mbox{ }

  This is a polymorphic function provided as a quick debugging aid.
  It is an identity function, which as a side-effect prints its
  argument to standard output exactly as it would be printed at
  top-level.  Output is flushed immediately.  For printing strings,
  the function {\tt print} is probably more useful than {\tt
    printVal}.

\item[{\tt printDepth :\ int ref}]\mbox{ }

  This variable determines the depth (in terms of nested constructors,
  records, tuples, lists, and vectors) to which values are printed by
  the top-level value printer and the function {\tt printVal}. The
  components of the value whose depth is greater than {\tt printDepth}
  are printed as `{\tt \#}'.  The initial value of {\tt printDepth} is
  20. 

\item[{\tt printLength :\ int ref}]\mbox{ }

  This variable determines the way in which list values are printed by
  the top-level value printer and the function {\tt printVal}. If the
  length of a list is greater than {\tt printLength}, only the first
  {\tt printLength} elements are printed, and the remaining elements
  are printed as `\ldots'.  The initial value of {\tt printLength} is
  200.

\item[{\tt quietdec :\ bool ref}]\mbox{ }

  This variable, when {\tt true}, turns off the interactive system's
  prompt and responses, except warnings and error messages.  Useful
  for writing scripts in SML\@.  The default value is {\tt false}; it
  can be set to {\tt true} with the {\tt -quietdec} command line
  option; see Section~\ref{sec-mosml-options}.

\item[{\tt quit :\ unit -> unit}]\mbox{ }

  Evaluating {\tt quit()} quits Moscow ML immediately.

\item[{\tt quotation :\ bool ref}]\mbox{ }

  Determines whether quotations and antiquotations are permitted in
  declarations entered at top-level and in files compiled with {\tt
    compile}; see Section~\ref{sec-quotations}.  When {\tt quotation}
  is {\tt false} (the default), the backquote character is an ordinary
  symbol which can be used in ML symbolic identifiers.  When {\tt
    quotation} is {\tt true}, the backquote character is illegal in
  symbolic identifiers, and a quotation {\tt `a b c`} will be
  evaluated to an object of type {\tt 'a frag list}.

\item[{\tt system :\ string -> int}]\mbox{ }

  Evaluating {\tt system "$com$"} causes the command $com$ to be
  executed by the operating system.  If a non-zero integer is
  returned, this must indicate that the operating system has failed to
  execute the command.  Under MS DOS, the integer returned always
  equals 0.

\item[{\tt use :\ string -> unit}]\mbox{ }

  Evaluating {\tt use "$f$"} causes ML declarations to be read from
  file $f$ as if they were entered from the console.  The file must be
  in the current directory or in a directory on the {\tt loadPath}
  list.  A file loaded by {\tt use} may, in turn, evaluate calls to
  {\tt use}.  For best results, use {\tt use} only at top level, or at
  top level within a {\tt use}'d file.

\item[{\tt valuepoly :\ bool ref}]\mbox{ }

  Determines whether the type checker should use `value polymorphism';
  see Section~\ref{sec-valuepoly}.  Command-line option {\tt
    -valuepoly} sets {\tt valuepoly} to {\tt true} (the default),
  whereas option {\tt -imptypes} sets {\tt valuepoly} to {\tt false};
  see Sections~\ref{sec-mosml-options} and~\ref{sec-mosmlc-options}.

\item[{\tt verbose :\ bool ref}]\mbox{ }

  Determines whether the signature inferred by a call to {\tt compile}
  will be printed.  The printed signature follows the syntax of Moscow
  ML signatures, so the output of {\tt compile "{\rm U.sml}"} can be
  edited to subsequently create file U.sig.  The default value is {\tt
    false}.
\end{description}


\section{Modules and compilation units}
\label{sec-compilation-units}


\subsection{Basic concepts}

A Moscow ML program can consist of one or more {\em compilation
  units\/}, or {\em units\/} for short.  A compilation unit consists
of an optional {\em unit signature\/} and a {\em unit body\/}.  The
unit signature specifies the contents of the unit; it is an interface
to the unit.  The unit body declares the contents of the unit; it
provides an implementation of the unit.  The following analogies may
be helpful:

\begin{center}
\begin{tabular}{@{}|l|ll|}\hline
Moscow ML       & unit signature &      unit body\\
Standard ML     & signature      &      structure\\
Caml Light      & module interface &    module implementation\\
Modula-2        & interface module &    implementation module\\\hline
\end{tabular}
\end{center}

\noindent The unit body is always present, whereas the signature can
be omitted.  When the unit signature is present, it is called the {\em
  explicit signature\/} to distinguish it from the signature inferred
when elaborating the unit body.  When present, the explicit signature
must be matched by the body, and only those identifiers specified in
the signature are visible outside the unit.  If no signature is given,
all identifiers visible at the end of the unit body are visible
outside the unit.

Units are closely associated with files, as in Modula-2.  The body of
the unit called `U' is defined in a file called `U.sml', and its
explicit signature (if any) in file `U.sig'.

\begin{center}
  \begin{tabular}{ccc}
\begin{tabular}{@{}|ll|}\hline
\multicolumn{2}{|c|}{Files containing program text:}\\\hline
\makebox[2cm][l]{U.sig} & unit signature, specifications\\
U.sml & unit body, declarations\\\hline
\end{tabular}
& &
\begin{tabular}{@{}|ll|}\hline
\multicolumn{2}{|c|}{Files created by the compiler:}\\\hline
U.ui & compiled unit signature \\
U.uo & compiled unit body, bytecode\\\hline
\end{tabular}
  \end{tabular}
\end{center}


\subsection{Units without explicit signature}
\label{sec-units-without}

A unit U without an explicit signature consists of a file U.sml
containing 

\begin{program}
structure U = struct {\rm \ldots\ declarations\ \ldots} end
\end{program}

\noindent This is the same as a simple SML structure declaration.
There must be no corresponding explicit signature file U.sig.


\subsection{Units with explicit signature}
\label{sec-units-with}

A unit U with an explicit signature consists of a signature file
U.sig containing

\begin{program}
signature U = sig {\rm \ldots\ specifications\ \ldots} end
\end{program}

\noindent and a file U.sml, containing

\begin{program}
structure U :> U = struct {\rm \ldots\ declarations\ \ldots} end
\end{program}

\noindent This is the same as a SML structure declaration with an
opaque signature constraint.  Note that the file name, signature name,
and structure name must be the same.  The notation `{\tt U :> U}' is
an opaque signature constraint, meaning that other units have no
access to the internals of U.sml, only to the signature U.sig.

To illustrate the difference between transparent and opaque signature
constraints, consider the Standard ML (not Moscow ML) declarations:

\begin{program}
signature SIG = sig                     structure S: SIG = struct
    type t                                  type t = int   
    val x: t                                val x = 17     
end;                                    end                             
\end{program}

\noindent Given these declarations, the expression {\tt S.x+33} will
typecheck.  Although the signature {\tt SIG} just says that there
exists a type {\tt t}, constraining {\tt S} with {\tt SIG} does not
hide the fact that {\tt S.x} is actually an integer.

On the other hand, an opaque signature constraint, as in Moscow ML
units, {\em does\/} hide the true nature of {\tt t} and {\tt x}:

\begin{program}
structure M :> SIG = struct
    type t = int
    val x = 17
end                             
\end{program}

\noindent After this declaration, {\tt M.x+33} would fail to
typecheck: the type checker cannot see that {\tt M.t} is {\tt int} and
{\tt M.x} is an integer.  Often such hiding is just what is needed for
software engineering purposes.


\subsection{Syntax of unit signatures}
\label{sec-syntax-signatures}

Moscow ML unit signatures are very similar to Standard ML signatures
as defined in \cite{Milner:1997:TheDefinition}; the differences are
explained below.  A unit signature (in file U.sig) has the form:

\begin{quot}
\begin{tabular}{@{}lcll}
{\it unitsig\/} & ::= 
  & {\tt signature} {\it unitid} {\tt = sig {\it uspec\/} end} 
                                                & named signature \\
& & {\it uspec\/}                               & signature (old syntax)\\[2ex]
{\it uspec} & ::= & {\tt val} {\it valdesc}     & value specification\\
& & {\tt type} {\it typdesc}                    & abstract type \\
& & {\tt type} {\it typbind}            & type abbreviation \\
& & {\tt eqtype} {\it typdesc}          & abstract equality type\\
& & {\tt datatype} {\it datbind}        & datatype\\
& & {\tt datatype} {\it datbind} {\tt withtype} {\it typbind\/} &
                                         datatype with typbind\\
& & {\tt exception} {\it exdesc}        & exception\\
& & {\tt local} {\it lspec} {\tt in} {\it uspec} {\tt end}
                                        & local specifications\\
& &                                     & empty\\
& & {\it uspec} $\langle${;}$\rangle$ {\it uspec}       & sequential 
%\\
%& & {\tt infix\ }  $\langle${\it d\/}$\rangle$ {\it id$_1$\/} $\cdots$\ 
%                                             {\it id$_n$}
%                                       & infix (L) directive\\
%& & {\tt infixr} $\langle${\it d\/}$\rangle$ {\it id$_1$\/} $\cdots$\ 
%                                             {\it id$_n$}
%                                       & infix (R) directive\\
%& & {\tt nonfix} {\it id$_1$\/} $\cdots$\ {\it id$_n$}
%                                        & nonfix directive
\\[2ex]

{\it lspec} & ::= & {\tt open} {\it unitid$_1$\/} $\cdots$\ {\it
  unitid$_n$\/}
                                        & open other units\\
& & {\tt type} {\it typbind}            & type abbreviation \\
& & {\tt local} {\it lspec\/} {\tt in} {\it lspec} {\tt end}
& local specifications\\
& &                                             & empty\\
& & {\it lspec\/} $\langle${;}$\rangle$ {\it lspec\/} & sequential\\
\end{tabular}
\end{quot}

\noindent Note:

\begin{enumerate}
\item Type abbreviations {\tt type {\it typbind\/}} are permitted in
  signatures.  

\item There are no structure specifications and no sharing
  specifications.

\item No type, value, or exception may be specified twice at
  top-level.

\item A {\tt local} specification can be used only to restrict the
  scope of {\tt open} specifications and type abbreviations.

\item An {\tt open} specification can appear only inside {\tt local}.

\item The `{\tt signature {\it unitid\/} = sig}' and `{\tt end}' parts
  may be left out, although this is not recommended.
\end{enumerate}

\noindent Restriction (2) is the most significant one.  Restriction
(3), and restrictions similar to (4) and (5), are imposed by the
Standard ML of New Jersey implementation also.


\subsection{Syntax of unit bodies}

A unit body (in file U.sml) has the form:

\begin{quot}
\begin{tabular}{@{}llll}
{\it unitbody\/} & ::= 
  & {\tt structure} {\it unitid} {\tt = struct {\it dec\/} end} 
                                                & structure \\
& & {\tt structure {\it unitid\/} :> {\it unitid\/} = struct {\it dec\/} end}
                                                & structure with
                                                signature \\
& & {\it dec\/}                                 & structure (old syntax)\\
\end{tabular}
\end{quot}

\noindent A long identifier can refer to entities declared in other
units.  In Moscow ML, the syntax of long identifiers is:

\begin{quot}
\begin{tabular}{@{}lcll}
{\it longid\/} & ::= &   \makebox[2.5cm][l]{{\it id\/}}  & identifier\\
               &     &   {\it unitid.id\/} & qualified identifier
\end{tabular}
\end{quot}

\noindent where {\it unitid\/} and {\it id\/} are arbitrary SML
identifiers (either symbolic or alphanumeric).

A qualified identifier {\it unitid.id\/} denotes an entity {\it id\/}
declared in the compilation unit {\it unitid\/}.  A qualified
identifier can denote either a value variable, a value constructor, an
exception constructor, or a type constructor.  As in Standard ML, a
{\it longid\/} appearing in a defining position, such as a value
variable in a pattern, cannot have a qualifier: the identifier being
defined will always belong to the current unit.

An {\tt open} declaration has the form

\begin{program}
open {\rm U}\sub{1} $\cdots$ {\rm U}\sub{n}
\end{program}

\noindent where U\sub{1} $\cdots$ U\sub{n} are names of units.  The
units are opened from left to right, in the order U\sub{1} $\cdots$ 
U\sub{n}.  The text following an {\tt open {\rm U}} declaration can
reference identifiers declared in U without explicitly specifying the
name of the unit, subject to the usual scope rules of Standard ML\@.
That is, one can use {\it id\/} instead of U.{\it id\/}.

In the interactive system, a unit must be loaded before it can be
opened.  In the batch compilation system, the linker links in (only)
the needed declarations from opened units.  

A unit body U.sml must elaborate to a structure S\@.  If there is an
explicit signature U.sig corresponding to U.sml, then the resulting
structure must match the explicit signature.  As in Standard ML (but
in contrast to Caml Light), no reference is made to the signature
while elaborating the unit body.


\subsection{An example program consisting of three units}
\label{sec-example-units}

To illustrate the module system, we present a tiny program working
with arithmetic expressions.  It consists of three units {\tt Expr},
{\tt Reduce}, and {\tt Evaluate}.  This example is in {\tt
  mosml/examples/manual}.

File {\tt Expr.sml} below contains structure {\tt Expr}, which defines
a datatype {\tt expr} for representing expressions and a function {\tt
  show} to display them.  It has no signature constraint and therefore
exports both the datatype and the function:

\begin{program}
structure Expr = struct
    datatype expr = Cst of int | Neg of expr | Plus of expr * expr
        
    fun show (Cst n)         = makestring n
      | show (Neg e)         = "(-" ^ show e ^ ")"
      | show (Plus (e1, e2)) = "(" ^ show e1 ^ "+" ^ show e2 ^ ")"
end
\end{program}

\noindent File {\tt Reduce.sig} below contains the signature {\tt
  Reduce}, which specifies a function for reducing expressions.  It
mentions the type {\tt Expr.expr} from {\tt Expr}:

\begin{program}
signature Reduce = sig
    val reduce : Expr.expr -> Expr.expr
end
\end{program}

\noindent File {\tt Reduce.sml} below contains the structure {\tt
  Reduce}, which has a signature constraint, and therefore exports
only the function {\tt reduce} specified in the signature:

\begin{program}
structure Reduce :> Reduce = struct
    local open Expr
    in
        fun negate (Neg e) = e
          | negate e       = Neg e
        fun reduce (Neg (Neg e))      = e
          | reduce (Neg e)            = negate (reduce e)
          | reduce (Plus (Cst 0, e2)) = reduce e2
          | reduce (Plus (e1, Cst 0)) = reduce e1
          | reduce (Plus (e1, e2))    = Plus (reduce e1, reduce e2)
          | reduce e                  = e
    end
end
\end{program}

\noindent File {\tt Evaluate.sig} below contains the signature {\tt
  Evaluate}, which specifies a function {\tt eval} for evaluating
expressions, and a function {\tt test}.  Note the use of `{\tt open
  Expr}' to make the type {\tt expr} refer to {\tt Expr.expr}:

\begin{program}
signature Evaluate = sig
    local open Expr 
    in 
        val eval : expr -> int
        val test : expr -> bool
    end
end
\end{program}

\noindent File {\tt Evaluate.sml} below contains structure {\tt
  Evaluate}, which has a signature constraint, and mentions unit {\tt
  Expr} as well as {\tt Reduce}:

\begin{program}
structure Evaluate :> Evaluate = struct 
    local open Expr 
    in 
        fun eval (Cst n)         = n
          | eval (Neg e)         = \verb#~# (eval e)
          | eval (Plus (e1, e2)) = eval e1 + eval e2;
        fun test e = (eval e = eval (Reduce.reduce e))
    end
end
\end{program}


\subsection{Compiling, linking, and loading units}

Units can be compiled and linked using the batch compiler {\tt
  mosmlc}; see Section~\ref{sec-batch-compiler}.  Units compiled with
option {\tt -c} can be linked together.  Use {\tt mosml -o mosmlout
  A.uo} to produce a linked executable bytecode file {\tt mosmlout}
which will invoke the runtime system {\tt camlrunm}.  Use {\tt mosml
  -noheader -o mosmlout A.uo} to produce a linked bytecode file which
can be executed by {\tt camlrunm mosmlout}.  The linker will
automatically link any required bytecode files into {\tt mosmlout}.
See Section~\ref{sec-mosmlc-options} for more options.

Units can also be compiled from and loaded into the interactive system
{\tt mosml} using the primitives {\tt compile} and {\tt load}; see
Section~\ref{sec-nonstandard-interactive} above.


\subsection{Organizing  programs for compatibility with SML Modules}
\label{sec-compatible-modules}

Moscow ML and Standard ML of New Jersey (version 110) implement the
same core language, and many of the same libraries.  Here we give
advice on organizing structures and signatures so that they can be
compiled by both systems.

Assume we have a software system consisting of three structures A, B,
and C, where A and B each have a signature constraint, but C does not.
Assume further that C depends on A and B\@.  (There must be no functors
or nested structures in A and B).  We organize them in five files:

\begin{quot}
\begin{tabular}{|l|l|}\hline
Source file & File contents\\\hline
A.sig   & \tt signature A = sig \ldots\ end\\
B.sig   & \tt signature B = sig \ldots\ end\\
A.sml   & \tt structure A :> A = struct \ldots\ end\\
B.sml   & \tt structure B :> B = struct \ldots\ end\\
C.sml   & \tt structure C = struct \ldots\ A.foo \ldots\ B.bar \ldots\ end\\\hline
\end{tabular}
\end{quot}

\noindent Now we can compile these files using {\tt mosmlc} and load
them into a {\tt mosml} session as follows (where `\verb#$#' is the
shell prompt and `\verb#-#' is the ML prompt):

\begin{program}
\$\ mosmlc -c A.sig B.sig A.sml B.sml C.sml
\$\ mosml
- load "C";
\end{program}

\noindent Or, we can load and compile them in an SML/NJ session as
follows:

\begin{program}
\$\ sml
- app use ["A.sig", "B.sig", "A.sml", "B.sml", "C.sml"];
\end{program}

\noindent Hence the same source files can be used unmodified in both
systems.  Note that in Moscow ML, {\tt mosmlc} will create bytecode
files A.ui, A.uo, and so on.  If the source files do not change, there
is no need to recompile them with {\tt mosmlc}; just reload them using
the fast {\tt load} function.

If the source files {\em do\/} change, and have to be recompiled at
every use, it may be more practical to use the function {\tt compile}:

\begin{program}
\$\ mosml
- app compile ["A.sig", "B.sig", "A.sml", "B.sml", "C.sml"];
- load "C";
\end{program}


\subsection{Matching a unit body against a signature}
\label{sec-matching-body-signature}

A unit body S matches a signature SIG under the conditions described
in the Definition of Standard ML \cite{Milner:1997:TheDefinition}.
Roughly, this means:

\begin{itemize}
\item a value specification {\tt val} {\it v:t\/} must be matched by a
  value variable or value constructor or exception constructor {\it
v\/} in S whose type generalizes {\it t\/}
 
\item a type abbreviation {\tt type} {\it t = ty\/} must be matched by
  the same type abbreviation {\it t = ty\/} in S

\item an abstract type {\it t\/} must be matched by some type {\it
    t\/} in S

\item an abstract equality type {\it t\/} must be matched by a type
  {\it t\/} in S admitting equality

\item a datatype must be matched by precisely the same datatype in S

\item an exception constructor {\it E\/} of type {\it t\/}
  must be matched by an exception constructor {\it E\/} in S whose
type generalizes {\it t\/}
\end{itemize}

\noindent Moreover, to facilitate separate compilation, there are some
representation constraints:

\begin{enumerate}
\item If the specified argument type of a value constructor (in a
  datatype specification) is an explicit tuple or record, then the
  declared argument type must be an explicit tuple or record also, and
  vice versa.  This restriction does not apply if there is only one
  constructor in the datatype.

\item The order of value constructors in a datatype specification must
  be the same as in the matching datatype declaration.
\end{enumerate}


\newpage

\section{The batch compiler}
\label{sec-batch-compiler}

Moscow ML includes a batch compiler {\tt mosmlc} in addition to the
interactive system {\tt mosml}.  It compiles and links programs
non-interactively, and can turn them into standalone executable files.
The batch compiler can be invoked from a Makefile, which simplifies
the (re)compilation of large programs considerably; see
Section~\ref{sec-recompilation-guide}.


\subsection{Overview}

The {\tt mosmlc} command has a command-line interface similar to that
of most C compilers.  It accepts several types of arguments: source
files for unit bodies, source files for unit signatures, and compiled
unit bodies.

\begin{itemize}
\item An argument ending in .sig is taken to be the name of a source
  file containing a unit signature.  Given a file U.sig, the compiler
  produces a compiled signature in the file U.ui.

\item An argument ending in .sml is taken to be the name of a source
  file containing a unit body.  Given a file U.sml, the compiler
  produces compiled object code in the file U.uo.  It also produces an
  inferred signature file U.ui if there is no explicit signature U.sig.

\item An argument ending in .uo is taken to be the name of a compiled
  unit body.  Such files are linked together, along with the compiled
  unit bodies obtained by compiling .sml arguments (if any), and the
  necessary Moscow ML library files, to produce a standalone
  executable program.  

  The linker automatically includes any additional bytecode files
  required by the files specified on the command line; option {\tt -i}
  makes it report all the files that were linked.  The linker issues a
  warning if a file {\tt B} is required by a file {\tt A} that
  precedes {\tt B} in the command line.  At run-time, the top-level
  declarations of the files are evaluated in the order in which the
  files were linked; in the absence of any warning, this is the order
  of the files on the command line.
\end{itemize}

\noindent The output of the linking phase is a file containing
compiled code that can be executed by the runtime system {\tt
  camlrunm}.  If {\tt mosmlout} is the name of the file produced by the
linking phase, the command

\begin{program}
camlrunm mosmlout arg\sub{1} arg\sub{2} \ldots\ arg\sub{n}
\end{program}

\noindent executes the compiled code contained in {\tt mosml.out}.
The list of arguments can be obtained in Moscow ML by evaluating
the expression {\tt CommandLine.arguments ()}.

\begin{description}
\item[MS Windows and DOS:] If the output file produced by the linking
  phase has extension {\tt .exe}, and option {\tt -noheader} is not
  used, then the file is directly executable.  Hence, an output file
  named {\tt mosmlout.exe} can be executed with the command

\begin{program}
mosmlout arg1 arg2 ... argn    
\end{program}

The output file {\tt mosmlout.exe} consists of a tiny executable file
prepended to a linked bytecode file.  The executable invokes the {\tt
  camlrunm} runtime system to interpret the bytecode.  As a
consequence, this is not a standalone executable: it still requires
{\tt camlrunm.exe} to reside in one of the directories in the path.

\item[Unix:] The output file produced by the linking phase is directly
  executable (unless the {\tt -noheader} option is used).  It
  automatically invokes the {\tt camlrunm} runtime system, either
  using a tiny executable prepended to the linked bytecode file, or
  using the Unix incantation \verb|#!/usr/local/bin/camlrunm| or
  similar.  In the former case, {\tt camlrunm} must be in one of the
  directories in the path; in the latter case it must be in {\tt
    /usr/local/bin}.  To create a true stand-alone executable you may
  simply concatenate the runtime system with the bytecode file
  produced by {\tt mosmlc -noheader}, but this adds 60--150 KB to the
  size of the executable, depending on your version of Unix:

\begin{program}
cat /usr/local/bin/camlrunm mosmlout > mosmlbin
chmod a+x mosmlbin
\end{program}
\end{description}


\subsection{Command-line options}
\label{sec-mosmlc-options}

The following command-line options are recognized by {\tt mosmlc}.

\begin{description}
\item[{\tt -c}]\mbox{ }

  Compile only. Suppresses the linking phase of the compilation.
  Source code files are turned into compiled files (.ui and .uo), but
  no executable file is produced.  This option is useful for compiling
  separate units.

\item[{\tt -files {\rm response-file}}]\mbox{ }

  Pass the names of files listed in file response-file to the linking
  phase just as if these names appeared on the command line.  File
  names in response-file are separated by blanks (spaces, tabs,
  newlines) and must end either in .sml or .uo.  A name U.sml
  appearing in the response file is equivalent to U.uo.  Use this
  option to overcome silly limitations on the length of the
  command line (as in MS DOS).

\item[{\tt -g}]\mbox{ }

  This option causes some information about exception names to be
  written at the end of the executable bytecode file.  

\item[{\tt -i}]\mbox{ }

  Causes the compiler to print the inferred signature of the unit body
  or bodies being compiled.  Also causes the linker to list all object
  files linked.  A U.sig file corresponding to a given U.sml file can
  be produced semi-automatically by piping the output of the compiler
  to a file U.out, and subsequently editing this file to obtain a file
  U.sig.

\item[{\tt -msgstyle \textrm{style}}]\mbox{ }
  
  By specifying \texttt{-msgstyle msdev}, one can make the compiler
  generate error messages understandable by Microsoft Developer
  Studio.  The default behaviour is to generate error messages
  understandable the Emacs editor in SML mode.

\item[{\tt -noautolink}]\mbox{ }

  In version 1.42 and later, the linker automatically links in any
  additional object files required by the files explicitly specified
  on the command line.  Option {\tt -noautolink} reinstates the
  behaviour of pre-1.42 versions: all object files must be explicitly
  specified in the appropriate order.

\item[{\tt -standalone}]\mbox{ }
  
  Unix: Specifies that the runtime system should be prepended to the
  linked bytecode, thus creating a stand-alone executable.  This adds
  75--100 KB to the size of the linked file.  MS Windows: This option
  cannot be used.  All mosmlc-generated executables require the
  dynamically linked library \texttt{camlrt.dll} to be present in the
  directory \verb+C:\mosml\bin+.

\item[{\tt -stdlib {\rm stdlib-directory}}]\mbox{ }

  Specifies the standard library directory, which will be searched by
  the compiler and linker for the .ui and .uo files corresponding to
  units mentioned in the files being linked.  The default standard
  library is set when the system is created, and is usually {\tt
    \$\{HOME\}/mosml/lib} under Unix and \verb#c:\mosml\lib# under MS
  Windows and DOS.

\item[{\tt -I {\rm directory}}]\mbox{ }

  Add the given directory to the list of directories searched for
  compiled signature files (.ui) and compiled object code files (.uo).
  By default, the current directory is searched first, then the
  standard library directory.  Directories added with {\tt -I} are
  searched after the current directory, but before the standard
  library directory.  When several directories are added with several
  {\tt -I} options on the command line, these directories are searched
  from left to right.

\item[{\tt -valuepoly}]\mbox{ }

  Specify that the type checker should use `value polymorphism'; see
  Section~\ref{sec-valuepoly}.  Default.

\item[{\tt -imptypes}]\mbox{ }

  Specify that the type checker should distinguish imperative and
  applicative type variables, generalize all applicative type
  variables, and generalize imperative type variables only in
  non-expansive expressions.  See Section~\ref{sec-valuepoly}.

\item[{\tt -o {\rm exec-file}}]\mbox{ }

  Specify the name of the output file produced by the linker.  In the
  absence of this option, a default name is used.  In MS Windows and
  DOS, the default name is {\tt mosmlout.exe}; in Unix it is {\tt
    a.out}.

\item[{\tt -P {\rm unit-set}}]\mbox{ }

  Determines which library units will be {\em open\/} at compile-time.
  Any library unit in the load path can be used by the compiler for
  type checking purposes.  Thus regardless of the {\tt -P} option, the
  compiler knows the type of library functions such as {\tt
    Array.foldl}.

  \begin{description}
  \item[{\tt -P default}] The units {\tt Char}, {\tt List}, and {\tt
      String} will be partially opened.  This is the default,
    permitting e.g.\ {\tt String.concat} to be referred to just as
    {\tt concat}.

  \item[{\tt -P sml90}] Provides an initial environment which is
    upwards compatible with that of the 1990 {\em Definition of
      Standard ML\/} and with pre-1.30 releases of Moscow ML\@.  In
    particular, the functions {\tt chr}, {\tt explode}, {\tt implode},
    and {\tt ord} work on strings, not characters.  The math functions
    and input-output facilities required by the 1990 Definition
    \cite[Appendix C and D]{Milner:1990:TheDefinition} are available
    at top-level.  In addition the same (new) libraries are opened as
    with {\tt -P default}.

  \item[{\tt -P nj93}] Provides a top-level environment which is
    mostly compatible with that of SML/NJ 0.93.  The functions {\tt
      app}, {\tt ceiling}, {\tt chr}, {\tt dec}, {\tt explode}, {\tt
      fold}, {\tt hd}, {\tt implode}, {\tt inc}, {\tt max}, {\tt min},
    {\tt nth}, {\tt nthtail}, {\tt ord}, {\tt ordof}, {\tt revapp},
    {\tt revfold}, {\tt substring}, {\tt tl}, and {\tt truncate} have
    the same type and meaning as in SML/NJ 0.93.  The math functions
    and input-output facilities required by the 1990 Definition
    \cite[Appendix C and D]{Milner:1990:TheDefinition} are available
    at top-level.  In addition the same (new) libraries are opened as
    with {\tt -P default}.  This option does {\em not\/} imply {\tt
      -imptypes}.

  \item[{\tt -P full}] Same as {\tt -P default}.

  \item[{\tt -P none}] No library units are initially opened.
  \end{description}
  
  Additional directories to be searched for library units can be
  specified with the {\tt -I {\rm directory}} option.

\item[{\tt -noheader}]\mbox{ }

  Causes the output file produced by the linker to contain only the
  bytecode, not preceded by any executable code.  A file {\tt
    mosmlout} thus obtained can be executed only by explicitly
  invoking the runtime system as follows: {\tt camlrunm mosmlout}.

\item[{\tt -q}]\mbox{ }

  Enables the quotation/antiquotation mechanism; see
  Section~\ref{sec-quotations}.

\item[{\tt -v}]\mbox{ }

 Prints the version number of the various passes of the
  compiler.
\end{description}


\section{Recompilation management}
\label{sec-recompilation-guide}

Recompilation management helps the programmer recompile only what is
necessary after a change to a unit signature or unit body.  

Type-safe linking prevents the programmer from creating unsafe or
meaningless programs.  The {\tt load} function and the batch linker
ensure probabilistically type-safe linking, so it is virtually
impossible to cause the system to create a type-unsafe program.  


\subsection{Using `make' to manage recompilation}
\label{sec-using-make}

Consider the example program in Section~\ref{sec-example-units}
consisting of the three units Evaluate, Expr, and Reduce.  Assume
their source files *.sig and *.sml reside in a particular directory.
Copy a Makefile stub (see below) to that directory, and change to that
directory.

\begin{enumerate}
\item Edit the Makefile so that the names of the bytecode files
  Evaluate.uo, Expr.uo, and Reduce.uo appear on the line beginning
  with `{\tt all:}' (see the example makefiles below).

\item Compute the dependencies among the files by executing:
\begin{program}
make depend
\end{program}

\item Recompile all those files which have not yet been compiled, or
  which have been modified but not yet recompiled, or which depend on
  modified files, by executing:
        
\begin{program}
make    
\end{program}
\end{enumerate}

\noindent Step (3) must be repeated whenever you have modified a
component of the program system.  Step (2) need only be repeated if
the inter-dependencies of some components change, or if you add or
remove an explicit signature file.  Step (1) need only be repeated
when you add or delete an entire unit of the program system.

Old versions of the compiled *.ui and *.uo files can be removed by
executing:

\begin{program}
make clean  
\end{program}

\noindent The inter-dependencies are computed by a small ML program
{\tt mosmldep}, which correctly handles nested comments and strings in
the source files.


\subsection{An example Makefile for Unix}
\label{sec-makefile-unix}

To use the Makefile below, first edit it so that all the required
units (.uo files) appear on the line beginning with `{\tt all:}', then
proceed as explained in Section~\ref{sec-using-make}.  You do not need
to edit any other part of the Makefile.  In particular, the
dependencies following {\tt DO NOT DELETE THIS LINE} are generated
automatically when executing {\tt make depend} (as above).  A copy of
the Makefile can be found in {\tt mosml/tools/Makefile.stub}.

You will need only the Unix utility {\tt make}.

\begin{program}
# Unix Makefile stub for separate compilation with Moscow ML.  

MOSMLHOME=\$\{HOME\}/mosml
MOSMLTOOLS=camlrunm \$(MOSMLHOME)/tools
MOSMLC=mosmlc -c
MOSMLL=mosmlc
MOSMLLEX=mosmllex
MOSMLYACC=mosmlyac

.SUFFIXES :
.SUFFIXES : .sig .sml .ui .uo

all: Evaluate.uo Expr.uo Reduce.uo

clean:
        rm -f *.ui
        rm -f *.uo
        rm -f Makefile.bak

.sig.ui:
        \$(MOSMLC) \$<

.sml.uo:
        \$(MOSMLC) \$<

depend: 
        rm -f Makefile.bak
        mv Makefile Makefile.bak
        \$(MOSMLTOOLS)/cutdeps < Makefile.bak > Makefile
        \$(MOSMLTOOLS)/mosmldep >> Makefile

### DO NOT DELETE THIS LINE
Evaluate.ui: Expr.uo 
Evaluate.uo: Evaluate.ui Expr.uo Reduce.ui 
Reduce.uo: Reduce.ui Expr.uo 
Reduce.ui: Expr.uo 
\end{program}


\subsection{An example Makefile for MS DOS}
\label{sec-makefile-dos}

To use the Makefile below, first edit it so that all the required
units (.uo files) appear on the line beginning with `{\tt all:}', then
proceed as explained in Section~\ref{sec-using-make}.  You do not need
to edit any other part of the Makefile.  In particular, the
dependencies following {\tt DO NOT DELETE THIS LINE} are generated
automatically when executing {\tt make depend} (as above).  A copy of
this makefile can be found in {\tt
  mosml\verb#\#tools\verb#\#makefile.stb}.

You will need a DOS version of {\tt make}, such as that from Borland
C++ version 2.0 or 3.0.

\begin{program}
# DOS Makefile stub for separate compilation with Moscow ML.  

MOSMLHOME=c:\verb#\#mosml
MOSMLTOOLS=camlrunm \$(MOSMLHOME)\verb#\#tools
MOSMLC=mosmlc -c
MOSMLL=mosmlc
MOSMLLEX=mosmllex
MOSMLYACC=mosmlyac

all: evaluate.uo expr.uo reduce.uo

clean:
        del *.ui
        del *.uo
        del makefile.bak

.sig.ui:
        \$(MOSMLC) \$<

.sml.uo:
        \$(MOSMLC) \$<

depend:
        del makefile.bak
        ren makefile makefile.bak
        \$(MOSMLTOOLS)\verb#\#cutdeps < makefile.bak > makefile
        \$(MOSMLTOOLS)\verb#\#mosmldep >> makefile

### DO NOT DELETE THIS LINE
evaluate.uo: evaluate.ui expr.uo reduce.ui 
reduce.ui: expr.uo 
reduce.uo: reduce.ui expr.uo 
evaluate.ui: expr.uo 
\end{program}


\subsection{Unit names and DOS file names}

Recompilation management for DOS is essentially as for Unix, except
for the usual complications that follow from the restrictions on the
length of file names, and from their case-insensitivity.

Under MS DOS, filenames are all the same case and can be at most 8
characters long (plus a 3 character extension).  Since file names are
used as unit names, this may cause problems.  We attempt to circumvent
these problems as follows:

\begin{itemize}
\item Unit names used inside ML programs under DOS are `normalized':
  the first character is made upper case (if it is a letter), all
  other characters are made lower case, and the unit name is truncated
  to eight characters.  Hence a unit which resides in file {\tt
    commands.sml} can be referred to as unit {\tt Commands} inside an
  ML program, and can also be referred to as {\tt CommandStructure},
  etc., since normalization transforms the latter into the former.

\item The following names are exceptions to this rule: {\tt BasicIO},
  {\tt BinIO}, {\tt CharArray}, {\tt CharVector},\\ {\tt CommandLine},
  {\tt FileSys}, {\tt ListPair}, {\tt OS}, {\tt StringCvt}, {\tt
    Substring}, {\tt TextIO}, {\tt Word8Array}, {\tt Word8Vector};
  they are normalized precisely as shown in this list.  This is to
  accommodate the SML Basis Library.

\item In DOS makefiles, the file names appearing after {\tt all:}\
  must be all lower case and at most 8 characters long (otherwise
  `make' will not work properly).  For instance, the unit {\tt
    CharArray} must be called {\tt chararra} in a DOS makefile.

\item A unit name given as argument to {\tt load}, to {\tt compile},
  or to the batch compiler, is truncated and made lower case by DOS as
  usual, so evaluating {\tt load "VeryLongName"} will load bytecode
  file {\tt verylong.uo}.
\end{itemize}

\newpage

\section{Value polymorphism}
\label{sec-valuepoly}

The 1997 revision of Standard ML \cite{Milner:1997:TheDefinition}
adopts value polymorphism, discarding the distinction between
imperative (\verb#'_a#) and applicative (\verb#'a#) type variables,
and generalizing type variables only in non-expansive expressions.
Consider a {\tt val}-binding

\begin{program}
val x = e;
\end{program}

\noindent With {\em value polymorphism\/}, the free type variables in
the type of {\tt x} are generalized only if the right-hand side {\tt
  e} is non-expansive.  This is a purely syntactic criterion: an
expression is {\em non-expansive\/} if it has the form {\it nexp\/},
defined by the grammar below:

\begin{quot} 
\begin{tabular}{@{}lcll}
{\it nexp} & ::= & {\it scon\/}  & special constant\\
&& {\it longid\/} & (possibly qualified) identifier\\           
&& \verb#{# $\langle${\it nexprow\/}$\rangle$ \verb#}# & record of
non-expansive expressions\\
&& \verb#(# {\it nexp\/} \verb#)# & parenthesized non-expansive expression\\
&& {\it con\/}\ {\it nexp\/} & constructor application, where {\it con\/}
is not {\tt ref}\\
&& {\it excon\/}\ {\it nexp\/} & exception constructor application\\
&& {\it nexp} \verb#:# {\it ty\/} & typed non-expansive expression\\
&& \verb#fn# {\it match} & function abstraction\\[2ex]

{\it nexprow} & ::= & {\it lab\/} \verb#=# {\it nexp\/}
$\langle$\verb#,# {\it nexprow\/}$\rangle$  \\
\end{tabular}
\end{quot}

\noindent Roughly, a non-expansive expression is just a value, that
is, an expression in normal form.  For example, the right-hand side
{\tt length} below is an identifier, and so is non-expansive.  Hence
the free type variable {\tt 'a} in the type {\tt 'a list -> int} of
{\tt x} becomes generalized:

\begin{program}
- val x = length;
> val x = fn : 'a list -> int 
\end{program}

\noindent On the other hand, the right-hand side {\tt (fn f => f)
  length} below, although it evaluates to the same value as the
previous one, is expansive: it is not derivable from the above
grammar.  Hence the type variable {\tt 'a} will not be generalized,
and type checking will fail:

\begin{program}
- val x = (fn f => f) length;
! Toplevel input:
! val x = (fn f => f) length;
! ^^^^^^^^^^^^^^^^^^^^^^^^^^
! Value polymorphism: Free type variable at top level
\end{program}

\noindent In Standard ML, all type variables in types reported at
top-level must be universally quantified; there must be no free type
variables.  When type checking fails for this reason, there are two
remedies: Either (1) insert a type constraint to eliminate the type
variables, or (2) {\em eta-expand\/} the right-hand side to make it
non-expansive:

\begin{program}
- val x1 = (fn f => f) length : bool list -> int;
> val x1 = fn : bool list -> int

- val x2 = fn ys => (fn f => f) length ys;
> val x2 = fn : 'a list -> int    
\end{program}

\noindent In Moscow ML versions prior to 1.40, the type checker would
distinguish imperative and applicative type variables, generalize all
applicative type variables, and generalize imperative type variables
only in non-expansive expressions, as required by the 1990 {\em
  Definition\/} \cite{Milner:1990:TheDefinition}.  To reinstate this
behaviour, invoke {\tt mosml} or {\tt mosmlc} with the option {\tt
  -imptypes}.  This is useful for compiling old programs.

\newpage


\section{Weak pointers}
\label{sec-weak-pointers}

Moscow ML supports weak pointers and arrays of weak pointers, using
library structure {\tt Weak}.  A {\em weak pointer\/} is a pointer
that cannot itself keep an object alive.  Hence the object pointed to
by a weak pointer may be deallocated by the garbage collector if the
object is reachable only by weak pointers.

The interface to arrays of weak pointers is the same as that of
standard arrays (structure {\tt Array}), but the subscript function
{\tt sub} may raise exception {\tt Fail} if the accessed object is
dead.  On the other hand, if {\tt sub} returns a value, it is
guaranteed not to die unexpectedly: it will be kept alive by the
returned pointer.  Also, the weak array iteration functions iterate
only over the live elements of the arrays.

One application of weak pointers is to implement hash consing without
space leaks.  The idea in hash consing is to re-use pairs: whenever a
new pair ($a$, $b$) is to be built, an auxiliary table is checked to
see whether such a pair exists already.  If so, the old pair is
reused.  In some applications, this may conserve much space and time.
However, there is a danger of running out of memory because of a space
leak: the pair ($a$, $b$) cannot be deallocated by the garbage
collector because it remains forever reachable from the auxiliary
table.  To circumvent this problem, one creates a weak pointer from
the auxiliary table to the pair, so that the auxiliary table in itself
cannot keep the pair alive.

For an example, see {\tt mosml/examples/weak}.  See also the {\tt
  Weak} signature; try `{\tt help "Weak";}'.


\section{Dynamic linking of foreign functions}
\label{sec-dynamic-linking}

Moscow ML supports dynamic linking of foreign (C) functions, using
library structure {\tt Dynlib}\footnote{Thanks to Ken Larsen.}.  A
library of functions may be written in C and compiled into a
dynamically loadable library, using appropriate compiler options.
With the {\tt Dynlib} structure one can load this library and call the
C functions from Moscow ML, without recompiling the runtime system.

It is the responsibility of the C functions to access and construct
SML values properly, using the macroes defined in {\tt
  mosml/src/runtime/mlvalues.h}.  For this reason, the foreign
function interface is included only with the source distribution.  As
usual, type or storage mistakes in C programs may crash your programs.

The ML garbage collector may run at any time an ML memory allocation
is made.  This may cause ML values to be moved (from the young
generation to the old one).  To make sure that ML heap pointers needed
by your C function are adjusted correctly by the garbage collector,
register them using the \verb+Push_roots+ and \verb+Pop_roots+ macroes
from {\tt runtime/memory.h}.

To modify a value in the ML heap, you must use the {\tt Modify} macro
from {\tt runtime/memory.h}; otherwise you may confuse the incremental
garbage collector and crash your program.

When loading the compiled library one must specify the absolute path
unless it has been installed as a system library.  This may require
putting it in a particular directory, such as {\tt /lib} or {\tt
  /usr/lib}, or editing {\tt /etc/ld.so.conf} and running {\tt
  ldconfig}.

To compile Moscow ML\footnote{Supported under Linux, FreeBSD, NetBSD,
  Solaris, Digital Unix, HP-UX, MacOS, and MS Windows'95/98/NT.} with
support for dynamic linking, edit file {\tt mosml/src/Makefile.inc} as
indicated there.

For more information, see the examples in directory {\tt
  mosml/src/dynlibs}, in particular\\ {\tt
  mosml/src/dynlibs/interface}.  See also the {\tt Dynlib} signature;
try `{\tt help "Dynlib";}'.


\section{Using GNU gdbm persistent hash tables }
\label{sec-using-mgdbm}

Moscow ML provides an interface to GNU gdbm persistent hashtables, via
structures {\tt Gdbm} and {\tt Polygdbm}; this requires {\tt Dynlib},
see Section~\ref{sec-dynamic-linking}.  GNU gdbm provides fast access
even to very large hashtables stored on disk, ensuring mutual
exclusion, handy for creating simple databases for use by CGI scripts
etc.

GNU gdbm (not included with Moscow ML) must be installed, and the
interface to GNU gdbm defined in {\tt mosml/src/dynlibs/mgdbm} must be
compiled and installed before
{\tt Gdbm} and {\tt Polygdbm} can be used.  For instructions, see file
{\tt mosml/src/dynlibs/mgdbm/README}.


\section{Using POSIX regular expressions}
\label{sec-using-mregex}

Moscow ML provides an interface to the GNU regex implementation of
POSIX 1003.2 regular expressions, with additional support for
replacing matching substrings etc., via structure {\tt Regex}\@.  This
requires {\tt Dynlib}, see Section~\ref{sec-dynamic-linking}.

The GNU regex library (which is included with Moscow ML) and the
interface defined in\\ {\tt mosml/src/dynlibs/mregex} must be compiled
and installed before
structure {\tt Regex}  can be used.  For instructions, see file
{\tt mosml/src/dynlibs/mregex/README}.


\section{Using the PostgreSQL relational database server}
\label{sec-using-mpq}

Moscow ML provides an interface to the PostgreSQL relational database
server, via structure {\tt Postgres}\@.  This requires {\tt Dynlib},
see Section~\ref{sec-dynamic-linking}.

The PostgreSQL database server (not included) must be installed, and
the interface to PostgreSQL defined in {\tt mosml/src/dynlibs/mpq}
must be compiled and installed before structure {\tt Postgres} can be
used.  For instructions, see file {\tt mosml/src/dynlibs/mpq/README}.


\section{Using the MySQL relational database server}
\label{sec-using-mmysql}

Moscow ML provides an interface to the MySQL relational database
server, via structure {\tt Mysql}\footnote{Thanks to Thomas S.
  Iversen.}\@.  This requires {\tt Dynlib}, see
Section~\ref{sec-dynamic-linking}.

The MySQL database server (not included) must be installed, and the
interface to Mysql defined in\\ {\tt mosml/src/dynlibs/mmysql} must be
compiled and installed before structure {\tt Mysql} can be used.  For
instructions, see file {\tt mosml/src/dynlibs/mmysql/README}.


\section{Using the GIF image library}
\label{sec-using-mgd}

Moscow ML provides an interface to Thomas Boutell's gd graphics
package for creating GIF images, via structure {\tt Gdimage}\@.  This
requires {\tt Dynlib}, see Section~\ref{sec-dynamic-linking}.

Thomas Boutell's gd image package (not included) must be installed,
and the interface defined in\\ {\tt mosml/src/dynlibs/mgd} must be
compiled and installed before structure {\tt Gdimage} can be used.
For instructions, see file {\tt mosml/src/dynlibs/mgd/README}.


\section{Using the Socket interface}
\label{sec-using-msocket}

Moscow ML provides an interface to Internet and file sockets, via
structure {\tt Socket}\footnote{Thanks to Ken Larsen; initial
  development financed by the PROSPER project.}, which adheres fairly
closely to the SML Basis Library structure of the same name.  Using
structure {\tt Socket} requires {\tt Dynlib}, see
Section~\ref{sec-dynamic-linking}.

The sockets interface defined in {\tt mosml/src/dynlibs/msocket} must
be compiled and installed before structure {\tt Socket} can be used.
For instructions, see file {\tt mosml/src/dynlibs/msocket/README}.

\newpage
\section{Quotations and antiquotations}
\label{sec-quotations}

Moscow ML implements {\em quotations\/}, a non-standard language
feature useful for embedding object language phrases in ML programs.
Quotations are disabled by default.  This feature originates in the
Standard ML of New Jersey implementation.  To enable quotations in the
interactive system ({\tt mosml}), execute {\tt quotation := true}.
This allows quotations to appear in declarations entered at top-level
and in files compiled by the primitive {\tt compile}.  To enable
quotations in files compiled with the batch compiler {\tt mosmlc},
invoke it with option {\tt -q} as in {\tt mosmlc -q}.

A quotation is a particular kind of expression and consists of a
non-empty sequence of (possibly empty) {\em fragments\/} surrounded by
backquotes:

\begin{quot} 
\begin{tabular}{@{}lcll}
{\it exp} & ::= & \makebox[5cm][l]{{\tt `}{\it frags\/}{\tt `}}  & quotation\\[2ex]

{\it frags} & ::= & {\it charseq\/} & character sequence \\
& & {\it charseq\/} {\tt \verb#^#}{\it id} {\it frags\/}
                                        & antiquotation variable\\
& & {\it charseq\/} {\tt \verb#^#(}{\it exp}{\tt )} {\it frags\/} 
                                        & antiquotation expression\\
\end{tabular}
\end{quot}

\noindent The {\it charseq\/} is a possibly empty sequence of
printable characters or spaces or tabs or newlines.  A quotation
evaluates to a value of type {\tt ty frag list} where {\tt ty} is the
type of the antiquotation variables and antiquotation expressions, and 
the type {\tt 'a frag} is defined as follows:

\begin{program}
datatype 'a frag = QUOTE of string | ANTIQUOTE of 'a
\end{program}

\noindent A {\it charseq\/} fragment evaluates to {\tt QUOTE "{\it
    charseq\/}"}.  An antiquotation fragment {\tt \verb#^#{\it id\/}}
or {\tt \verb#^#(}{\it exp\/}{\tt )} evaluates to {\tt ANTIQUOTE {\it
    value\/}} where {\it value\/} is the value of the variable {\it
  id\/} resp.\ the expression {\it exp\/}.  All antiquotations in a
quotation must have the same type {\tt ty}.

An antiquotation fragment is always surrounded by (possibly empty)
quotation fragments; and no two quotation fragments can be adjacent.
The entire quotation is parsed before any antiquotation inside it is
evaluated.  Hence changing the value of {\tt Meta.quotation} in an
antiquotation inside a quotation has no effect on the parsing of the
containing quotation.

For an example, say we have written an ML program to analyse C program
phrases, and that we want to enter the C declaration {\tt char s[6] =
  "abcde"}.  We could simply define it as a string:

\begin{program}
val phrase = "char s[6] = \verb#\"#abcde\verb#\#"";
\end{program}

\noindent but then we need to escape the quotes ({\tt "}) in the C
declaration, which is tiresome.  If instead we use a quotation, these
escapes are not needed:

\begin{program}
val phrase = `char s[6] = "abcde"`;
\end{program}

\noindent It evaluates to {\tt [QUOTE "char s[6] =
  \verb#\"#abcde\verb#\#""] :\ 'a frag list}.  Moreover, suppose we
want to generate such declarations for other strings than just {\tt
  "abcde"}, and that we have an abstract syntax for C phrases:

\begin{program}
datatype cprog =
    IntCst of int
  | StrCst of string
  | ...  
\end{program}

\noindent Then we may replace the string {\tt "abcde"} by an
antiquotation {\tt \verb#^#(StrCst str)}, and the array dimension 6 by
an antiquotation {\tt \verb#^#(IntCst (size str + 1))}, and make the
string {\tt str} a function parameter:

\begin{program}
fun mkphrase str = `char s[^(IntCst (size str + 1))] = ^(StrCst str)`;
\end{program}

\noindent Evaluating {\tt mkphrase "longer"} produces the following
representation of a C phrase:

\begin{program}
[QUOTE "char s[", ANTIQUOTE (IntCst 7), QUOTE "] = ", 
 ANTIQUOTE (StrCst "longer"), QUOTE ""] : cprog frag list
\end{program}


\newpage

\section{A lexer generator}
\label{sec-mosmllex}

This section describes {\tt mosmllex}, a lexer generator which is
closely based on {\tt camllex} from the Caml Light implementation by
Xavier Leroy.  This documentation is based on that of {\tt camllex}
also.  


\subsection{Overview}

Given a set of regular expressions with attached semantic actions,
{\tt mosmllex} produces a lexical analyser in the style of {\tt lex}.
If file {\tt lexer.lex} contains a specification of a lexical
analyser, then executing

\begin{program}
mosmllex lexer.lex  
\end{program}

\noindent produces a file {\tt lexer.sml} containing Moscow ML code
for the lexical analyser.  This file defines one lexing function per
entry point in the lexer definition.  These functions have the same
names as the entry points.  Lexing functions take as argument a lexer
buffer, and return the semantic attribute of the corresponding entry
point.

Lexer buffers are an abstract data type implemented in the library
unit {\tt Lexing}.  The functions\\ {\tt createLexerString} and {\tt
  createLexer} from unit {\tt Lexing} create lexer buffers that read
from a character string, or any reading function, respectively.

When used in conjunction with a parser generated by {\tt mosmlyac}
(see Section~\ref{sec-mosmlyac}), the semantic actions compute a value
belonging to the datatype {\tt token} defined by the generated parsing
unit.

Example uses of {\tt mosmllex} can be found in directories {\tt calc}
and {\tt lexyacc} under {\tt mosml/examples}.


\subsection{Hints on using {\tt mosmllex}}

A lexer definition must have a rule to recognize the special symbol
{\tt eof}, meaning end-of-file.  In general, a lexer must be able to
handle all characters that can appear in the input.  This is usually
achieved by putting the wildcard case \verb#_# at the very end of the
lexer definition.  If the lexer is to be used with e.g.\ MS Windows,
MS DOS or MacOS files, remember to provide a rule for the
carriage-return symbol \verb#\r#.  Most often \verb#\r# will be
treated the same as \verb#\n#, e.g.\ as whitespace.

Do not use string constants to define many keywords; this may produce
large lexer programs.  It is better to let the lexer scan keywords the
same way as identifiers and then use an auxiliary function to
distinguish between them.  For an example, see the {\tt keyword}
function in {\tt mosml/examples/lexyacc/Lexer.lex}.


\subsection{Syntax of lexer definitions}

The format of a lexer definition is as follows:

\begin{program}
\verb#{# {\rm header} \verb#}#
let {\rm abbrev} = {\rm regexp}
\ldots
let {\rm abbrev} = {\rm regexp}
rule {\rm entrypoint} =
  parse {\rm regexp} \verb#{# {\rm action} \verb#}#
      | \ldots
      | {\rm regexp} \verb#{# {\rm action} \verb#}#
and {\rm entrypoint} =
  parse \ldots
and \ldots
;
\end{program}

\noindent Comments are delimited by {\tt (*} and {\tt *)}, as in
SML\@.  An abbreviation (abbrev) for a regular expression may refer
only to abbreviations that strictly precede it in the list of
abbreviations; in particular, abbreviations cannot be recursive.


\subsubsection{Header}

The header section is arbitrary Moscow ML text enclosed in curly
braces \verb#{# and \verb#}#.  It can be omitted.  If it is present,
the enclosed text is copied as is at the beginning of the output file
{\tt lexer.sml}.  Typically, the header section contains the {\tt
  open} directives required by the actions, and possibly some
auxiliary functions used in the actions.

\subsubsection{Entry points}

The names of the entry points must be valid ML identifiers.

\subsubsection{Regular expressions}

The regular expressions regexp are in the style of {\tt lex}, but with
a more ML-like syntax.

\begin{description}
\item[{\tt `{\rm char}`}]\mbox{ }

  A character constant, with a syntax similar to that of Moscow ML
  character constants; see Section~\ref{sec-character-constants}.
  Match the denoted character.

\item[{\tt \_}]\mbox{ }

   Match any character.

\item[{\tt eof}]\mbox{ }

  Match the end of the lexer input.

\item[{\tt "{\rm string}"}]\mbox{ }

  A string constant, with a syntax similar to that of Moscow ML string
  constants; see Section~\ref{sec-string-constants}.  Match the
  denoted string.

\item[{\tt [ {\rm character-set} ]}]\mbox{ }

  Match any single character belonging to the given character set.
  Valid character sets are: single character constants {\tt `c`};
  ranges of characters {\tt `$c_1$` - `$c_2$`} (all characters between
  $c_1$ and $c_2$, inclusive); and the union of two or more character
  sets, denoted by concatenation.

\item{{\tt [ \verb#^# {\rm character-set} ]}}\mbox{ }

  Match any single character not belonging to the given character set.

\item[{\tt {\rm regexp} *}]\mbox{ }

  Match the concatenation of zero or more strings that match regexp.
  (Repetition).

\item[{\tt {\rm regexp} +}]\mbox{ }

  Match the concatenation of one or more strings that match regexp.
  (Positive repetition).

\item[{\tt {\rm regexp} ?}]\mbox{ }

  Match either the empty string, or a string matching regexp.
  (Option).

\item[{\tt {\rm regexp}$_1$ | {\rm regexp}$_2$}]\mbox{ }

  Match any string that matches either regexp$_1$ or regexp$_2$.
  (Alternative).

\item[{\tt {\rm regexp}$_1$ {\rm regexp}$_2$}]\mbox{ }

  Match the concatenation of two strings, the first matching
  regexp$_1$, the second matching regexp$_2$.  (Concatenation).

\item[{\tt {\rm abbrev} }]\mbox{ }

  Match the same strings as the regexp in the most recent {\tt
    let}-binding of abbrev.

\item[{\tt ( {\rm regexp} )}]\mbox{ }

  Match the same strings as regexp.

\end{description}

\noindent The operators {\tt *} and {\tt +} have highest precedence,
followed by {\tt ?}, then concatenation, then {\tt |} (alternative).


\subsubsection{Actions}

An action is an arbitrary Moscow ML expression.  An action is
evaluated in a context where the identifier {\tt lexbuf} is bound to
the current lexer buffer.  Some typical uses of {\tt lexbuf} in
conjunction with the operations on lexer buffers (provided by the {\tt
  Lexing} library unit) are listed below.

\begin{description}
\item[{\tt Lexing.getLexeme lexbuf}]\mbox{ }

  Return the matched string.

\item[{\tt Lexing.getLexemeChar lexbuf n}]\mbox{ }
 
  Return the {\tt n}'th character in the matched string.  The first
  character has number 0.

\item[{\tt Lexing.getLexemeStart lexbuf}]\mbox{ }

  Return the absolute position in the input text of the beginning of
  the matched string.  The first character read from the input text
  has position 0.

\item[{\tt Lexing.getLexemeEnd lexbuf}]\mbox{ }

  Return the absolute position in the input text of the end of the
  matched string.  The first character read from the input text has
  position 0.

\item[{\tt entrypoint lexbuf}]\mbox{ }

  Here {\tt entrypoint} is the name of another entry point in the same
  lexer definition.  Recursively call the lexer on the given entry
  point.  Useful for lexing nested comments, for example.
\end{description}


\subsubsection{Character constants}
\label{sec-character-constants}

A character constant in the lexer definition is delimited by \verb#`#
(backquote) characters.  The two backquotes enclose either a space or
a printable character {\it c\/}, different from \verb#`# and \verb#\#,
or an escape sequence:

\begin{quot}
\begin{tabular}{@{}|l|l|}\hline
Sequence & Character denoted \\\hline
\verb#`#{\it c\/}\verb#`# & the character {\it c\/}\\
\verb#`\\`#        &  backslash (\verb#\#)\\
\verb#`\``#               & backquote (\verb#`#)\\
\verb#`\n`#       & newline (LF)\\
\verb#`\r`#       & return (CR)\\
\verb#`\t`#       & horizontal tabulation (TAB)\\
\verb#`\b`#       & backspace (BS)\\
\verb#`\^#{\it c\/}\verb#`#      & the ASCII character control-{\it c\/}\\
\verb#`\#{\it ddd\/}\verb#`#     & the character with ASCII code {\it ddd\/} in
decimal\\\hline 
\end{tabular}
\end{quot}

\subsubsection{String constants}
\label{sec-string-constants}

A string constant is a (possibly empty) sequence of characters
delimited by {\tt "} (double quote) characters.

\begin{quot}
\begin{tabular}{lcll}
  {\it string-literal\/} & ::= & {\tt "}{\it strcharseq\/}{\tt "}
        & non-empty string\\
  & &                            {\tt ""} & empty string\\[2ex]

{\it strcharseq\/} & ::= & {\it strchar\/} $\langle${\it
  strcharseq\/}$\rangle$ & character sequence
\end{tabular}
\end{quot}

\noindent A string character {\it strchar\/} is a space, or a
printable character {\it c\/} (except {\tt "} and \verb#\#), or an
escape sequence:

\begin{quot}
\begin{tabular}{@{}|l|l|}\hline
Sequence & Character denoted \\\hline
{\it c\/}       & the character {\it c\/}\\
\verb#\\#       & backslash (\verb#\#)\\
\verb#\"#       & double quote (\verb#"#)\\
\verb#\n#       & newline (LF)\\
\verb#\r#       & return (CR)\\
\verb#\t#       & horizontal tabulation (TAB)\\
\verb#\b#       & backspace (BS)\\
\verb#\^#{\it c\/}      & the ASCII character control-{\it c\/}\\
\verb#\#{\it ddd\/}     & the character with ASCII code {\it ddd\/} in
decimal\\\hline 
\end{tabular}
\end{quot}

\newpage

\section{A parser generator}
\label{sec-mosmlyac}

This section describes {\tt mosmlyac}, a simple parser generator which
is closely based on {\tt camlyacc} from the Caml Light implementation
by Xavier Leroy; {\tt camlyacc} in turn is based on Bob Corbett's
public domain Berkeley {\tt yacc}.  This documentation is based on
that in the Caml Light reference manual.


\subsection{Overview}

Given a context-free grammar specification with attached semantic
actions, {\tt mosmlyac} produces a parser, in the style of {\tt yacc}.
If file {\tt grammar.grm} contains a grammar specification, then
executing

\begin{program}
mosmlyac grammar.grm  
\end{program}

\noindent produces a file {\tt grammar.sml} containing a Moscow ML
unit with code for a parser and a file {\tt grammar.sig} containing
its interface.

The generated unit defines a parsing function {\tt S} for each start
symbol {\tt S} declared in the grammar.  Each parsing function takes
as arguments a lexical analyser (a function from lexer buffers to
tokens) and a lexer buffer, and returns the semantic attribute of the
corresponding entry point.  Lexical analyser functions are usually
generated from a lexer specification by the {\tt mosmllex} program.
Lexer buffers are an abstract data type implemented in the library
unit {\tt Lexing}.  Tokens are values from the datatype {\tt token},
defined in the signature file {\tt grammar.sig} produced by running
{\tt mosmlyac}.

Example uses of {\tt mosmlyac} can be found in directories {\tt calc}
and {\tt lexyacc} under {\tt mosml/examples}.


\subsection{The format of grammar definitions}

\begin{program}
\%\verb#{#
  {\rm header}
\%\verb#}#
  {\rm declarations}
\%\%
  {\rm rules}
\%\%
  {\rm trailer}
\end{program}

\noindent Comments in the declarations and rules sections are enclosed
in C comment delimiters {\tt /*} and {\tt */}, whereas comments in the
header and trailer sections are enclosed in ML comment delimiters {\tt
  (*} and {\tt *)}.


\subsubsection{Header and trailer}

Any SML code in the header is copied to the beginning of file {\tt
  grammar.sml}, after the {\tt token} datatype declaration; it usually
contains {\tt open} declarations required by the semantic actions of
the rules.  Any SML code in the trailer is copied to the end of file
{\tt grammar.sml}.  Both sections are optional.


\subsubsection{Declarations}

Declarations are given one per line.  They all start with a \verb#%# sign.

\begin{description}
\item[{\tt \%token {\rm symbol \ldots\ symbol}}]\mbox{ }

  Declare the given symbols as tokens (terminal symbols).  These
  symbols become constructors (without arguments) in the {\tt token}
  datatype.

\item[{\tt \%token < {\rm type} >  {\rm symbol \ldots\ symbol}}]\mbox{
    }

  Declare the given symbols as tokens with an attached attribute of
  the given type.  These symbols become constructors (with arguments
  of the given type) in the {\tt token} datatype.  The type part is an
  arbitrary Moscow ML type expression, but all type constructor names
  must be fully qualified (e.g.\ {\tt Unitname.typename}) for all
  types except standard built-in types, even if the proper {\tt open}
  declarations (e.g.\ {\tt open Unitname}) were given in the header
  section.  

\item[{\tt \%start {\rm symbol}}]\mbox{ }

  Declare the given symbol as entry point for the grammar.  For each
  entry point, a parsing function with the same name is defined in the
  output file {\tt grammar.sml}.  Non-terminals that are not declared
  as entry points have no such parsing function.  

\item[{\tt \%type < {\rm type} >  {\rm symbol \ldots\ symbol}}]\mbox{
    }

  Specify the type of the semantic attributes for the given symbols.
  Every non-terminal symbol, including the start symbols, must have
  the type of its semantic attribute declared this way.  This ensures
  that the generated parser is type-safe.  The type part may be an
  arbitrary Moscow ML type expression, but all type constructor names
  must be fully qualified (e.g.\ {\tt Unitname.typename}) for all
  types except standard built-in types, even if the proper {\tt open}
  declaration (e.g.\ {\tt open Unitname}) were given in the header
  section.  

\item[{\tt \%left {\rm symbol \ldots\ symbol}}]\mbox{ }
\item[{\tt \%right {\rm symbol \ldots\ symbol}}]\mbox{ }
\item[{\tt \%nonassoc {\rm symbol \ldots\ symbol}}]\mbox{ }

  Declare the precedence and associativity of the given symbols.  All
  symbols on the same line are given the same precedence.  They have
  higher precedence than symbols declared in previous {\tt \%left},
  {\tt \%right} or {\tt \%nonassoc} lines.  They have lower precedence
  than symbols declared in subsequent {\tt \%left}, {\tt \%right} or
  {\tt \%nonassoc} lines.  The symbols are declared to associate to
  the left ({\tt \%left}), to the right ({\tt \%right}), or to be
  non-associative ({\tt \%nonassoc}).  The symbols are usually tokens,
  but can also be dummy nonterminals, for use with the {\tt \%prec}
  directive inside the rules.
\end{description}


\subsubsection{The format of grammar rules}

\begin{program}
{\rm nonterminal} :
    {\rm symbol \ldots\ symbol} \verb#{# {\rm semantic-action} \verb#}#
  | \ldots
  | {\rm symbol \ldots\ symbol} \verb#{# {\rm semantic-action} \verb#}#
;
\end{program}

\noindent Each right-hand side consists of a (possibly empty) sequence
of symbols, followed by a semantic action.  

The directive `{\tt \%prec} symbol' may occur among the symbols in a
rule right-hand side, to specify that the rule has the same precedence
and associativity as the given symbol.

Semantic actions are arbitrary Moscow ML expressions, which are
evaluated to produce the semantic attribute attached to the defined
nonterminal.  The semantic actions can access the semantic attributes
of the symbols in the right-hand side of the rule with the \verb#$#
notation: \verb#$1# is the attribute of the first (leftmost) symbol,
\verb#$2# is the attribute of the second symbol, etc.  An empty
semantic action evaluates to {\tt () :\ unit}.

Actions occurring in the middle of rules are not supported.  Error
recovery is not implemented.


\subsection{Command-line options of {\tt mosmlyac}}
\label{sec-mosmlyac-options}

\begin{description}
\item[{\tt -v}]\mbox{ }

  Generate a description of the parsing tables and a report on
  conflicts resulting from ambiguities in the grammar.  The
  description is put in file {\tt grammar.output}.

\item[{\tt -b{\rm prefix}}]\mbox{ }

  Name the output files {\tt prefix.sml}, {\tt prefix.sig}, {\tt
    prefix.output}, instead of using the default naming convention.
\end{description}


\subsection{Reporting lexer and parser errors}

Lexical errors (e.g.\ illegal symbols) and syntax errors can be
reported in an intelligible way by using the {\tt Location} module
from the Moscow ML library.  It provides functions to print out
fragments of a source text, using location information from the lexer
and parser.  See {\tt help "Location.sig"} for more information.  See
file {\tt mosml/examples/lexyacc/Main.sml} for an example.

\newpage

\section{Copyright and credits}
\label{sec-copyright}

\begin{description}
\item[Copyright notice] Moscow ML - a lightweight implementation of
  Standard ML\@.  Copyright (C) 1994, 1995, 1996, 1997, 1998, 1999,
  2000.  Sergei Romanenko, Moscow, Russia and Peter Sestoft,
  Copenhagen, Denmark.

This program is free software; you can redistribute it and/or
modify it under the terms of the GNU General Public License
as published by the Free Software Foundation; either version 2
of the License, or (at your option) any later version.

This program is distributed in the hope that it will be useful,
but WITHOUT ANY WARRANTY; without even the implied warranty of
MERCHANTABILITY or FITNESS FOR A PARTICULAR PURPOSE.  See the
GNU General Public License for more details.

You should have received a copy of the GNU General Public License
along with this program; if not, write to the Free Software
Foundation, Inc., 675 Mass Ave, Cambridge, MA 02139, USA.

Note that a number of source files are derived from the Caml Light
distribution, copyright (C) 1993 INRIA, Rocquencourt, France.
Thus charging money for redistributing Moscow ML may require prior
permission from INRIA; see the INRIA copyright notice in file
copyrght/copyrght.cl.\\[2ex]


\item[Main Moscow ML contributors] 
  \begin{itemize}
  \item Doug Currie (e@flavors.com), Flavors Technology, USA.
  \item Sergei Romanenko (roman@keldysh.ru), Keldysh Institute of
    Applied Mathematics, Russian Academy of Sciences, Russia.
  \item Claudio V.\ Russo (Claudio.Russo@cl.cam.ac.uk),
    University of Cambridge.  Much of the work was done at the
    University of Edinburgh under project XXXXXXX??.
  \item Peter Sestoft (sestoft@dina.kvl.dk), Department of Mathematics
    and Physics, Royal Veterinary and Agricultural University, Denmark
    and The IT University of Copenhagen, Denmark.  Part of the work
    was done while at the Department of Computer Science at the
    Technical University of Denmark, and while visiting AT\&T 
    Bell Laboratories, Murray Hill, New Jersey, USA.
  \end{itemize}

\item[Moscow ML owes much to]\mbox{ }
\begin{itemize}
\item the CAML Light implementation by Xavier Leroy and Damien
  Doligez (INRIA, Rocquencourt, France);
\item the ML Kit by Lars Birkedal, Martin Elsman, Niels Hallenberg,
  Nick Rothwell, Mads Tofte and David Turner (Copenhagen University,
  Denmark, and Edinburgh University, Scotland);
\item inspiration from the SML/NJ compiler developed at Princeton
  University and AT\&T Bell Laboratories, New Jersey, USA; and
\item feedback, suggestions, contributions, or proposals for new
  features from Jonas Barklund, Michael Gordon, Ken Friis Larsen,
  Michael Norrish, Konrad Slind, and numerous other people.
\end{itemize}
\end{description}


\section{How to get Moscow ML}
\label{sec-how-to-get}

\begin{itemize}
\item The Moscow ML home page is
        http://www.dina.kvl.dk/\verb#~#sestoft/mosml.html
\item The Linux executables are in
        ftp://ftp.dina.kvl.dk/pub/mosml/linux-mos20bin.tar.gz
\item The MS Windows executables are in 
        ftp://ftp.dina.kvl.dk/pub/mosml/win32-mos20bin.zip
\item The MS DOS executables are in 
        ftp://ftp.dina.kvl.dk/pub/mosml/mos20bin.zip
\item The Macintosh/MacOS (68k and PPC) executables are in\\
        ftp://ftp.dina.kvl.dk/pub/mosml/mac-mos20bin.sea.hqx
\item The DOS source files are in 
        ftp://ftp.dina.kvl.dk/pub/mosml/mos20src.zip
\item The Unix and MS Windows source files are in 
        ftp://ftp.dina.kvl.dk/pub/mosml/mos20src.tar.gz
\item The MacOS modified source files (relative to Unix) are in\\
        ftp://ftp.dina.kvl.dk/pub/mosml/mac-mos20src.sea.hqx
\end{itemize}

%You will need gcc, make, and perl to recompile Moscow ML for Unix.
%You will need djgpp, perl, and Borland C++ version 2.0 (or later) to
%recompile Moscow MLs for DOS\@.  Djgpp and go32 are found at
%ftp://oak.oakland.edu/pub/simtelnet/gnu/djgpp/ or
%ftp://ftp.sunet.se/pub/pc/mirror/simtelnet/gnu/djgpp/ or any other
%SimTel mirror site.  You will need MetroWERKS to recompile Moscow ML
%for the Macintosh.


\section{Books and other materials on Standard ML}
\label{sec-books}

\subsubsection*{The Definition and Commentary}

\begin{itemize}
\item Robin Milner, Mads Tofte and Robert Harper, {\em The Definition of
  Standard ML\/}, MIT Press 1990, ISBN 0-262-63132-6.  
\item Robin Milner, Mads Tofte, Robert Harper, and David B. MacQueen,
  {\em The Definition of Standard ML (Revised)\/}, MIT Press 1997,
  ISBN 0-262-63181-4.
\item Robin Milner and Mads Tofte, {\em Commentary on Standard ML\/},
  MIT Press 1991, ISBN 0-262-63137-7.
\end{itemize}

\subsubsection*{Textbooks available from publishers}

\begin{itemize}
\item Richard Bosworth, {\em A Practical Course in Functional
    Programming Using Standard ML}, McGraw-Hill 1995, ISBN
  0-07-707625-7.
\item Michael R. Hansen and Hans Rischel, \emph{Introduction to
    Programming using SML}, Addison-Wesley 1999, ISBN 0-201-39820-6.
\item Greg Michaelson, {\em Elementary Standard ML\/}, UCL Press 1995,
  ISBN 1-85728-398-8.
\item Colin Myers, Chris Clack, and Ellen Poon, {\em Programming with
    Standard ML\/}, Prentice Hall 1993, ISBN 0-13-722075-8.
\item Lawrence C. Paulson, {\em ML for the Working Programmer\/},
  Second edition.  Cambridge University Press 1996, ISBN
  0-521-56543-X.  
\item Chris Reade, {\em Elements of Functional Programming\/},
  Addison-Wesley 1989, ISBN 0-201-12915-9.
\item Ryan Stansifer, {\em ML Primer\/}, Prentice Hall 1992, ISBN
  0-13-561721-9.
\item Jeffrey D. Ullman, {\em Elements of ML Programming\/}, Prentice
  Hall 1994, ISBN 0-13-184854-2.
\item {\AA}ke Wikstr{\"o}m, {\em Functional Programming Using Standard
    ML\/}, Prentice Hall 1987, ISBN 0-13-331661-0.
\end{itemize}

\subsubsection*{Texts available on the net}

\begin{itemize}
\item Emden Gansner and John Reppy (editors): Standard ML Basis
  Library, hypertext version:\\ 
  \verb#http://www.cs.bell-labs.com/~jhr/sml/basis/index.html# \\
  \verb#http://www.dina.kvl.dk/~sestoft/sml/sml-std-basis.html#
  (mirror site)
  
\item Stephen Gilmore, \emph{Programming in Standard ML'97}, report
  ECS-LFCS-97-364, University of Edinburgh.  At
  http://www.dcs.ed.ac.uk/lfcsreps/EXPORT/97/ECS-LFCS-97-364/
  
\item Robert Harper, {\em Introduction to Standard ML\/}, report
  ECS-LFCS-86-14, University of Edinburgh, November 1986 (revised
  1989).  At
  ftp://ftp.cs.cmu.edu/afs/cs/project/fox/mosaic/intro-notes.ps.

\item Mads Tofte, {\em Tutorial on Standard ML\/}, Technical Report
  91/18, DIKU, University of Copenhagen, December 1991.  At 
  ftp://ftp.diku.dk/pub/diku/users/tofte/FPCA-Tutorial/
\end{itemize}

\bibliographystyle{plain}
\bibliography{books,mosml}

\end{document}
